\documentclass{article}
\usepackage[UTF8]{ctex} 
\usepackage{amsmath}
\usepackage{bm}
\usepackage{amssymb}
\usepackage{geometry}
\usepackage{graphicx}
\usepackage{listings}
\usepackage{tikz}
\usetikzlibrary{arrows.meta, positioning, shapes.geometric}
\usetikzlibrary{decorations.pathreplacing, fit}
\geometry{a4paper, margin=1.5cm}

\title{动态规划(Dynamic Programming, DP)}
\author{Tan Yiqing}
\date{\today}

\begin{document}
\maketitle
    
    \begin{figure}[h]
        \centering
        \includegraphics[width=0.7\textwidth]{"D:/program/data_construction/firefly/v2-b0e32140f47096599999f841ba7f09f4_r.jpg"}
    \end{figure}

\section{特点}
动态规划作为算法的重要基础之一,它代表一类算法,其主要特点包括:
\begin{itemize}
    \item 问题可以分解为较小规划的同类问题。
    \item 子问题多次重复出现。
\end{itemize}
它的核心概念是将大问题分成多个小问题,然后解决各个小问题,每个小问题只解决一次。

\section{实现方法}
动态规划通常有两种实现方法:
\begin{itemize}
    \item 从上到下:和递归类似,不同的是存储中间计算的结果,避免多次解决同一个小问题。
    \item 从下到上:把有关的子问题一个一个解决,最后计算出要的结果。
\end{itemize}

\section{解题思路}
动态规划的解题思路一般包括以下几个步骤:
\begin{itemize}
    \item 确定dp数组(dp table)及下标的含义。(这里的dp就是子问题的结果)
    \item 确定递推公式。
    \item 初始化dp数组。(一步一步初始化,防止数组越界)
    \item 确定遍历顺序。
    \item 举例推导dp数组。
\end{itemize}

\section{debug方法}
找问题的最好方式就是把dp数组打印出来,看一下究竟是不是按照自己的思路推导的。\\
做动态规划的题目,写代码之前一定要把状态转移在dp数组中的具体情况模拟一遍做到心中有数,确定最后推导出的结果是什么。
然后编写相应的代码,如果代码没通过就打印dp数组,看看和自己预先推导的哪里不一样。\\
如果打印结果和自己预先模拟推导的结果是一样的,那么就是自己的递推公式、dp数组的初始化或者遍历顺序有问题了。
如果打印结果和自己预先模拟推导的结果不一样,那么就是代码的实现细节有问题。

\section{例子}
\subsection{斐波那契数列}
很简单,主要是用于实践五个步骤。
\begin{enumerate}
    \item 确定dp数组及下标的含义:dp[i]表示第i个斐波那契数。
    \item 确定递推公式:dp[i] = dp[i-1] + dp[i-2]
    \item 初始化dp数组:dp[0] = 0, dp[1] = 1
    \item 确定遍历顺序:从小到大遍历。
    \item 举例推导dp数组:写出前十项:[0, 1, 1, 2, 3, 5, 8, 13, 21, 34]
\end{enumerate}
C++代码实现:
\begin{lstlisting}[language=C++]
int fib(int n) {
    if (n <= 1) return n;
    vector<int> dp(n + 1);
    dp[0] = 0;
    dp[1] = 1;
    for (int i = 2; i <= n; ++i) {
        dp[i] = dp[i - 1] + dp[i - 2];
    }
    return dp[n];
}
\end{lstlisting}
事实上我们没必要记录整个dp数组,只需要记录前两个数即可,空间复杂度可以优化到O(1)。
\begin{lstlisting}[language=C++]
int fib(int n) {
    if (n <= 1) return n;
    int a = 0, b = 1;
    for (int i = 2; i <= n; ++i) {
        int c = a + b;
        a = b;
        b = c;    //滚动数组,只维护两个数字,不存储整个数组。
    }
    return b;
}
\end{lstlisting}
当然这个也可以用从


\subsection{最长递增子序列}
很暴力的算法就是找出所有的子序列,再看哪些递增,再看哪个最长,时间复杂度是指数级别的。
动态规划的思路是:






\subsection{找零问题}
这个是作业题,所以就不写了


\end{document}