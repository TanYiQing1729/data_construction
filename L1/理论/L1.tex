\documentclass{article}
\usepackage[UTF8]{ctex} 
\usepackage{amsmath}
\usepackage{bm}
\usepackage{amssymb}
\usepackage{geometry}
\usepackage{graphicx}
\usepackage{listings}
\geometry{a4paper, margin=1.5cm}

\title{绪论}
\author{Tan Yiqing}
\date{\today}

\begin{document}
\maketitle

    \begin{figure}[h]
        \centering
        \includegraphics[width=0.70\textwidth]{"D:/program/data_construction/firefly/f0b0742bab8a4dd059b1f0c2eecdfa3e.jpg"}
    \end{figure}

\section{问题求解和程序设计}



\section{数据结构的基本概念}
\subsection{数据结构}
数据、数据元素、数据项、数据结构、逻辑结构、储存结构


\section{算法的基本概念}
\subsection{算法的定义}
算法是对特定问题求解过程的一种描述,为指令的有限序列。
\subsection{算法的特性}
有穷性,确定性,可行性,输入,输出,基本操作。  \\
好代码的特点:正确性,健壮性,可理解性,抽象分级,高效性。
\subsection{算法的描述方法}
自然语言,伪代码,流程图,程序语言。


\subsection{算法分析}
研究方法:事后统计,事前分析
    \subsubsection{时间复杂度}
    认为基本语句的时间一致,关心基本语句执行次数。\\
    \textbf{定义:} 对于两个函数 $f(n)$ 和 $g(n)$,如果存在正常数 $c$ 和 $n_0$,使得:
    \[
        \forall n > n_0,\, f(n) \leq c g(n)
    \]
    则称 $f(n)$ 的渐近上界为 $O(g(n))$,记作 $f(n) = O(g(n))$。
\begin{enumerate}
    \item \textbf{O($\log_2 n$) 示例:二分查找}
\begin{lstlisting}[language=C++]
int binarySearch(int arr[], int n, int target) {
    int left = 0, right = n - 1;
    while (left <= right) {
        int mid = left + (right - left) / 2;
        if (arr[mid] == target) return mid;
        else if (arr[mid] < target) left = mid + 1;
        else right = mid - 1;
    }
    return -1;
}
\end{lstlisting}

    \item \textbf{O($n$) 示例:线性查找}
\begin{lstlisting}[language=C++]
int linearSearch(int arr[], int n, int target) {
    for (int i = 0; i < n; ++i) {
        if (arr[i] == target) return i;
    }
    return -1;
}
\end{lstlisting}

    \item \textbf{O($n^2$) 示例:冒泡排序}
\begin{lstlisting}[language=C++]
void bubbleSort(int arr[], int n) {
    for (int i = 0; i < n - 1; ++i) {
        for (int j = 0; j < n - i - 1; ++j) {
            if (arr[j] > arr[j + 1]) {
                swap(arr[j], arr[j + 1]);
            }
        }
    }
}
\end{lstlisting}

    \item \textbf{O($2^n$) 示例:斐波那契数列}
\begin{lstlisting}[language=C++]
int fib(int n) {
    if (n <= 1) return n;
    return fib(n - 1) + fib(n - 2);
}
\end{lstlisting}

    \item \textbf{O($n!$) 示例:全排列}
\begin{lstlisting}[language=C++]
void permute(vector<int>& nums, int l, int r) {
    if (l == r) {
        // print permutation
        return;
    }
    for (int i = l; i <= r; ++i) {
        swap(nums[l], nums[i]);
        permute(nums, l + 1, r);
        swap(nums[l], nums[i]);
    }
}
\end{lstlisting}

\end{enumerate}
    
\begin{enumerate}
    \item \textbf{辗转相除法时间复杂度证明:}

设欧几里得算法为:对 $a>b>0$,不断执行 $(a,b)\leftarrow(b,\,a\bmod b)$ 直到 $b=0$。记余数序列
\[
r_{-1}=a,\quad r_0=b,\quad r_{i-1}=q_i\,r_i+r_{i+1}\ (q_i\in\mathbb{Z}_{\ge1}),\quad 0\le r_{i+1}<r_i.
\]
算法在 $r_t=0$ 时结束,步数为 $t$。

\textbf{关键不等式(两步减半):}
由 $r_i=q_{i+1}r_{i+1}+r_{i+2}$ 且 $q_{i+1}\ge1$,得
\[
r_i \ge r_{i+1}+r_{i+2}\ \ge\ 2r_{i+2}\quad\Rightarrow\quad r_{i+2}\ \le\ \frac{r_i}{2}\qquad(\forall i\ge0).
\]
因此对任意 $k\ge0$,
\[
r_{2k}\ \le\ \frac{r_0}{2^{k}}=\frac{b}{2^{k}},\qquad
r_{2k+1}\ \le\ \frac{r_1}{2^{k}}\ \le\ \frac{a}{2^{k}}.
\]

\textbf{步数上界:}
当 $k>\log_2 b$ 时,$r_{2k}<1$,而 $r_{2k}$ 为非负整数,只能为 $0$,算法已在(或早于)第 $2k$ 步结束。
故
\[
t\ \le\ 2\big\lceil \log_2 b\big\rceil\ =\ O(\log b)\ \subseteq\ O(\log n),\quad(n=\min\{a,b\}).
\]
结论:欧几里得算法时间复杂度为 $O(\log n)$。

\end{enumerate}


\section{数据结构}
\indent 定义:数据结构由数据与结构组成。数据之间的关系即为其结构。\\
结构的视点分为逻辑结构与存储结构,前者为数据之间的逻辑关系(如前面有看过的例子);
后者又称物理结构,为数据及逻辑结构在计算机中的表示。\\
\indent 逻辑结构是用户视角,面向问题的。存储结构是实现视角,面向计算机的。\\
\indent 数据类型:一组值的集合以及定义于这个集上的一组操作的总称。(例:整数类型)\\
\indent 抽象:抽象出本质的特征而略过细节。\\
\indent 抽象数据类型 (Abstract DataType, ADT):一个数据结构及对该结构上的一组操作的总称。\\




\begin{figure}
    \centering
    \includegraphics[width=0.5\textwidth]{D:/program/data_construction/L1/理论/ScreenShot_2025-10-31_144235_421.png}
\end{figure}

\end{document}