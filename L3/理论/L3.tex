\documentclass{article}
\usepackage[UTF8]{ctex} 
\usepackage{amsmath}
\usepackage{bm}
\usepackage{amssymb}
\usepackage{geometry}
\usepackage{graphicx}
\usepackage{listings}
\geometry{a4paper, margin=1.5cm}

\title{线性表特殊例子:栈和队列}
\author{Tan Yiqing}
\date{\today}

\begin{document}
\maketitle
    
    \begin{figure}[h]
        \centering
        \includegraphics[width=0.75\textwidth]{"D:/program/data_construction/firefly/v2-7904daf31705e9ec03cb510e40b9f3fa_r.jpg"}
    \end{figure}

\section{栈的逻辑结构}
\subsection{定义}
\indent 栈是只允许在固定一端进行插入和删除操作的线性表。
允许插入和删除的一端称为栈顶,另一端称为栈底。\\
\indent 栈的插入操作称为进栈(push),删除操作称为出栈(pop)。
栈中最后插入的元素最先被删除,称为后进先出(LIFO,Last In First Out)原则。\\

\subsection{出栈序列的数量}
\indent 设栈中有n个元素,按照1到n的顺序进栈(注:只要求顺序符合,不一定一口气进完),则所有可能的出栈序列共有$\frac{1}{n+1}\binom{2n}{n}$种。\\
\indent 例如:n=3时,共有5种出栈序列,分别为123,132,213,231,321。312没有的原因是要想弹出3,必须先把1和2压入栈中,可是要想输出1,必须输出2,矛盾。\\
\indent 这一现象与Catalan数有关,我们将它与二叉树建立双射关系来证明。\\
\indent 因此递推公式为:
\[
C_{n} = \sum_{i=0}^{n-1} C_{i} \cdot C_{n-1-i}
\]

\subsection{常见函数}

\subsubsection{构造函数}
\begin{itemize}
\item \textbf{前置条件}:栈不存在
\item \textbf{输入}:无
\item \textbf{功能}:栈的初始化
\item \textbf{输出}:无
\item \textbf{后置条件}:栈存在且为空
\end{itemize}

\subsubsection{析构函数}
\begin{itemize}
\item \textbf{前置条件}:栈存在
\item \textbf{输入}:无 
\item \textbf{功能}:销毁栈,释放动态申请的空间
\item \textbf{输出}:无
\item \textbf{后置条件}:栈不存在
\end{itemize}

\subsubsection{判断长度}
\begin{itemize}
\item \textbf{前置条件}:栈存在
\item \textbf{输入}:无
\item \textbf{功能}:回传栈的长度
\item \textbf{输出}:栈的长度
\item \textbf{后置条件}:栈没有变化
\end{itemize}

\subsubsection{判空}
\begin{itemize}
\item \textbf{前置条件}:栈存在
\item \textbf{输入}:无
\item \textbf{功能}:判断栈是否为空
\item \textbf{输出}:如功能
\item \textbf{后置条件}:栈没有变化
\end{itemize}

\subsubsection{进栈}
\begin{itemize}
\item \textbf{前置条件}:栈存在
\item \textbf{输入}:待插入元素x
\item \textbf{功能}:将元素插入栈顶
\item \textbf{输出}:无
\item \textbf{后置条件}:栈顶指针上移,栈长度加1
\end{itemize}

\subsubsection{出栈}
\begin{itemize}
\item \textbf{前置条件}:栈存在
\item \textbf{输入}:无
\item \textbf{功能}:将栈顶元素弹出。若栈为空则报错
\item \textbf{输出}:栈顶元素
\item \textbf{后置条件}:栈顶指针下移,栈长度减1,若栈为空则报错
\end{itemize}

\subsubsection{取栈顶元素}
\begin{itemize}
\item \textbf{前置条件}:栈存在
\item \textbf{输入}:无
\item \textbf{功能}:获取栈顶元素。若栈为空则报错
\item \textbf{输出}:栈顶元素
\item \textbf{后置条件}:栈没有变化
\end{itemize}

\section{栈的实现}
用数组实现,与向量表很像,关键指标last指向栈顶元素,top指向栈顶元素的下一个位置。

\subsection{顺序结构}
\subsubsection{入栈}
last右移,插入数据

\subsubsection{出栈}
last左移,回传数据,注意空栈的情况($last == -1$)

\subsubsection{其他注意点}
\begin{itemize}
\item 注意栈满的情况,但是时间复杂度有变化。
\item 两个栈共享空间,一个栈从数组左端开始,一个栈从数组右端开始。 
\end{itemize}

\subsection{链式结构}
这次不使用空结点,而是直接使用链表的头结点作为栈顶元素。
所有插入都在开始(头插法),删除也在开始。
操作基本上与单链表的头插法和头删法一样。空表为栈顶指针指向空白。




\section{队列的逻辑结构}
\subsection{定义}
\indent 队列是只允许在一端进行插入操作,而在另一端进行删除操作的线性表。
允许插入的一端称为队尾,允许删除的一端称为队头。\\
\indent 队列的插入操作称为入队(enqueue),删除操作称为出队(dequeue)。
队列中最先插入的元素最先被删除,称为先进先出(FIFO,First In First Out)原则。\\

\subsection{常见函数}

\subsubsection{构造函数}
\begin{itemize}
\item \textbf{前置条件}:队列不存在
\item \textbf{输入}:无
\item \textbf{功能}:队列的初始化
\item \textbf{输出}:无
\item \textbf{后置条件}:队列存在且为空
\end{itemize}

\subsubsection{析构函数}
\begin{itemize}
\item \textbf{前置条件}:队列存在
\item \textbf{输入}:无
\item \textbf{功能}:销毁队列,释放动态申请的空间
\item \textbf{输出}:无
\item \textbf{后置条件}:队列不存在
\end{itemize}

\subsubsection{判断长度}
\begin{itemize}
\item \textbf{前置条件}:队列存在
\item \textbf{输入}:无
\item \textbf{功能}:回传队列的长度
\item \textbf{输出}:队列的长度
\item \textbf{后置条件}:队列没有变化
\end{itemize}

\subsubsection{判空}
\begin{itemize}
\item \textbf{前置条件}:队列存在
\item \textbf{输入}:无
\item \textbf{功能}:判断队列是否为空
\item \textbf{输出}:如功能
\item \textbf{后置条件}:队列没有变化
\end{itemize}

\subsubsection{入队}
\begin{itemize}
\item \textbf{前置条件}:队列存在
\item \textbf{输入}:待插入元素x
\item \textbf{功能}:将元素插入队尾
\item \textbf{输出}:无
\item \textbf{后置条件}:队尾指针后移,队列长度加1
\end{itemize}

\subsubsection{出队}
\begin{itemize}
\item \textbf{前置条件}:队列存在
\item \textbf{输入}:无
\item \textbf{功能}:将队头元素弹出。若队列为空则报错
\item \textbf{输出}:队头元素
\item \textbf{后置条件}:队头指针后移,队列长度减1,若队列为空则报错
\end{itemize}

\subsubsection{取队头元素}
\begin{itemize}
\item \textbf{前置条件}:队列存在
\item \textbf{输入}:无
\item \textbf{功能}:获取队头元素。若队列为空则报错
\item \textbf{输出}:队头元素
\item \textbf{后置条件}:队列没有变化
\end{itemize}

\section{队列的实现}
基于数组实现,关键是有两个指标front和last,分别指向队头和队尾。
\subsubsection{入队}
last右移,插入数据

\subsubsection{出队}
front右移,回传数据,注意空队列的情况($front == last+1$)

\subsection{链式结构}
\indent 这用的更多一些,规避了数组容量的问题。这里还是使用空节点。
出队在队头的方向,跟单链表删除首个元素的一样。
如果删除唯一存在的数据,尾指针需要调整位置。
入队利用尾指针,插入到尾节点的后面,然后更新尾指针。
空队就是头指针和尾指针都指向同一个节点空节点。

\section{栈的一个简单应用:四则运算计算器}
\indent 使用RPN(Reverse Polish Notation)逆波兰表达式来计算。示例:\\
\indent 中缀表达式:1+2*3\\
\indent 逆波兰表达式:1 2 3 * +\\   
\indent 计算过程:如果是数字,入栈,如果是运算符,弹栈两次,计算后入栈。\\
\indent 例如计算1 2 3 * +的过程为:\\
\begin{itemize}
\item 读入1,压栈
\item 读入2,压栈
\item 读入3,压栈
\item 读入*,弹出3和2,计算2*3=6,压栈
\item 读入+,弹出6和1,计算1+6=7,压栈
\end{itemize}
最终结果为7。

\end{document}