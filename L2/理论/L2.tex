\documentclass{article}
\usepackage[UTF8]{ctex} 
\usepackage{amsmath}
\usepackage{bm}
\usepackage{amssymb}
\usepackage{geometry}
\usepackage{graphicx}
\usepackage{listings}
\geometry{a4paper, margin=1.5cm}

\title{线性表}
\author{Tan Yiqing}
\date{\today}

\begin{document}
\maketitle
    
    \begin{figure}[h]
        \centering
        \includegraphics[width=0.85\textwidth]{"D:/program/data_construction/firefly/_20250909224817_161_13.jpg"}
    \end{figure}

\section{线性表的逻辑结构}
\subsection{定义}
\indent 线性表,亦简称表,是 n 个句有相同类型的数据元素的有限序列。
\indent 规定线性表的长度为表中元素个数$n$,$n=0$时称为空表。
写作:$L={(a_1,a_2,...,a_n)}$,$a_i$为表中第i个元素,1$\leq$i$\leq$n。
空表写作:$L={()}$。\\

\subsection{特点}
\indent 有限性:线性表中元素个数有限。\\
\indent 相同性:线性表中元素类型相同。\\
\indent 顺序性:线性表中元素是有序排列的。\\ 

\subsection{功能}
\subsubsection{构造函数}
\begin{itemize}
\item \textbf{前置条件}:表不存在
\item \textbf{输入}:无
\item \textbf{功能}:表的初始化
\item \textbf{输出}:无
\item \textbf{后置条件}:表存在且为空
\end{itemize}

\subsubsection{析构函数}
\begin{itemize}
\item \textbf{前置条件}:表存在
\item \textbf{输入}:无
\item \textbf{功能}:销毁表,释放动态申请的空间
\item \textbf{输出}:无
\item \textbf{后置条件}:已释放所有在这个表的操作申请过的动态空间
\end{itemize}

\subsubsection{长度}
\begin{itemize}
\item \textbf{前置条件}:表存在
\item \textbf{输入}:无
\item \textbf{功能}:回传表的长度
\item \textbf{输出}:表的长度
\item \textbf{后置条件}:表没有变化
\end{itemize}

\subsubsection{是否空表}
\begin{itemize}
\item \textbf{前置条件}:表存在
\item \textbf{输入}:无
\item \textbf{功能}:判断表是否为空(若是空表则返回true,否则返回false)
\item \textbf{输出}:如功能
\item \textbf{后置条件}:表没有变化
\end{itemize}

\subsubsection{插入元素}
\begin{itemize}
\item \textbf{前置条件}:表存在
\item \textbf{输入}:元素值x,插入位置i
\item \textbf{功能}:把元素x插入到表的第i个位置。如$L={(a,b,c)}$,$i=1$,则插入后$L={(a,x,b,c)}$。
\item \textbf{输出}:若插入失败则抛出异常
\item \textbf{后置条件}:若插入成功,表的第i个位置为x,表的长度加1。
\end{itemize}

\subsubsection{删除元素}
\begin{itemize}
\item \textbf{前置条件}:表存在
\item \textbf{输入}:整数i
\item \textbf{功能}:把表的第i个位置的元素删除。如$L={(a,b,c)}$,$i=1$,则删除后$L={(a,c)}$。
\item \textbf{输出}:若删除成功,回传删掉的元素值;若删除失败则抛出异常
\item \textbf{后置条件}:若删除成功,表中元素减少一个,表的长度减1。
\end{itemize}

\subsubsection{查找所在位置的元素}
\begin{itemize}
\item \textbf{前置条件}:表存在
\item \textbf{输入}:整数i
\item \textbf{功能}:回传表中第i个位置的元素。如$L={(a,b,c)}$,$i=1$,则返回$b$。
\item \textbf{输出}:若i合法,回传表中第i个位置的元素;否则抛出异常
\item \textbf{后置条件}:表没有变化
\end{itemize}

\subsubsection{查找元素所在位置}
\begin{itemize}
\item \textbf{前置条件}:表存在
\item \textbf{输入}:元素值x
\item \textbf{功能}:查找元素x在表中的首次出现的位置;若不存在,回传-1。
\item \textbf{输出}:如功能
\item \textbf{后置条件}:表没有变化
\end{itemize}

\section{顺序表}
\subsection{定义}
\indent 顺序表的存储结构是用一个数列存储所有数据,另外用一个整数变量存储表的长度。
这是一个连续的存储结构,无需特别写出前驱后继关系。\\
\indent 可以认为是一维数组加上一堆功能。\\
\indent 注:需要数列的长度,如长度为1000,在插入第1001个元素时可能会出问题。\\

\subsection{实现}
\subsubsection{初始化、销毁、长度、是否空表}
\begin{itemize}
\item \textbf{初始化}:申请一个长度为maxSize的数组,length=0
\item \textbf{销毁}:释放数组空间,这里因为没东西,所以啥都不用做。
\item \textbf{长度}:返回length
\item \textbf{是否空表}:判断length是否为0
注:上述时间复杂度均为O(1)
\end{itemize}

\subsubsection{插入元素}
\begin{itemize}
\item 先把${(a_i,...,a_{n-1})}$后移一位,然后把$x$放到$a_i$的位置上,n++。
注:时间复杂度为O(n),与位置i和表长n有关,具体来说是$n-i+1$,平均复杂度(认为i是均匀分布)为O(n/2)=O(n)。
如果要追求效率应该让i尽量大。\\
有些i是非法的,比如i<0或i>n。
\end{itemize}

\subsubsection{删除元素}
\begin{itemize}
\item 先把${(a_{i+1},...,a_{n-1})}$前移一位,然后n--。
注:时间复杂度为O(n),与位置i和表长n有关,具体来说是$n-i$,平均复杂度(认为i是均匀分布)为O(n/2)=O(n)。
如果要追求效率应该让i尽量小。\\
有些i是非法的,比如i<0或i>n-1。而且$a_{n-1}$的位置的值是不变的。
\end{itemize}

\subsubsection{查找}
\begin{itemize}
\item \textbf{查找所在位置的元素}:当位置合理,直接返回$a_i$;否则抛出异常。时间复杂度为O(1)。
\item \textbf{查找元素所在位置}:从前往后遍历数组,找到第一个等于x的元素,返回其位置;否则返回-1。时间复杂度为O(n)。
\end{itemize}

\subsection{优缺点}
\subsubsection{优点}
\begin{itemize}
\item 不用特别处理元素的前驱后继关系,逻辑简单。
\item 随机存储,时间复杂度为O(1)。
\item 插入和删除操作相对简单,时间复杂度为O(n)。
\end{itemize}

\subsubsection{缺点}
\begin{itemize}
\item 容量限制,需要预先分配内存,可能导致空间浪费。
\item 插入和删除操作需要移动大量元素,效率较低。、
\item 碎片化问题,频繁的插入和删除操作可能导致内存碎片化,影响性能。
\end{itemize}

\section{向量表}
\subsection{定义}
\begin{itemize}
\item 向量表是顺序表的一种改进形式,基本操作和顺序表相同,但解决了顺序表容量限制的问题。\\
\item 数列是动态申请的,初始化时首次申请数列的空间;因此销毁时要释放空间。\\
\item 需要一个新的变量来储存目前表的容量。\\
\item 主要的区别在于插入的处理:当表满时,当表满时,申请一个新的数列空间(一般为原容量的双倍),
复制所有数据到新数列,然后把旧的数列空间释放。
\end{itemize}

\section{链表}
\subsection{定义}
\indent 数据的物理存储不一定按顺序,要想办法表示数据的前驱后继。
于是我们对每个数据同时配上若干个指针来达到目的。
单链表就是我们给每个数据配上一个指针,指向它的后继,最后一个指向空白。\\

\subsection{实现}
\subsubsection{是否空表,遍历}
\begin{itemize}
\item \textbf{是否空表}:判断首指针是否为空。
\item \textbf{遍历}:从头指针开始,依次访问每个节点,直到遍历完整个链表,指针指向空白。
\end{itemize}

\subsubsection{插入元素}
\begin{itemize}
\item \textbf{空表}:做一个新的节点,其指针指向空白,放入数据,首指针指向新的节点。
\item \textbf{非空表}:找到第i-1个节点,做一个新的节点,其指针指向第i个节点,放入数据,然后把第i-1个节点的指针指向新的节点。时间复杂度为O(n),与位置i有关,平均复杂度为O(n/2)=O(n)。
注:自己想想一下框框图。
\end{itemize}

\subsection{构造与销毁}
\begin{itemize}
\item \textbf{构造函数}:申请一个头节点,指针指向空白。长度设为0。
\item \textbf{析构函数}:从头节点开始,依次释放每个节点,直到指针指向空白。
具体来说是:暂存首指针指向的节点至K,首指针指向K的后继,释放K,重复直到首指针指向空白。
\end{itemize}

\subsection{长度}
\begin{itemize}
\item 在没有长度的私有变量的情况下。从头节点开始,依次访问每个节点,直到遍历完整个链表,指针指向空白。时间复杂度为O(n)。
\end{itemize}

\subsection{插入}
\begin{itemize}
\item 把数据x插入第i个位置。
\item 遍历到第i-1个节点,记为N。
\item 做一个新的节点,其指针指向第i个节点,放入数据,然后把第i-1个节点的指针指向新的节点。
注:时间复杂度为O(n),与位置i有关,平均复杂度为O(n/2)=O(n)。
\end{itemize}

\subsection{删除}
\begin{itemize}
\item 把第i个节点删除。
\item 遍历到第i-1个节点,记为N。
\item 用一个节点指针K暂存N的指针指向的节点(即第i个节点)。
\item 把N的指针指向K指向的下一个节点,即第i个节点的后继,释放第i个节点。
注:时间复杂度为O(n),与位置i有关,平均复杂度为O(n/2)=O(n)。
\end{itemize}

\subsection{查找}
\begin{itemize}
\item 每个节点遍历,直到找到第i个节点,返回其数据。
\end{itemize}

\section{题外话:算法设计}
\begin{itemize}
    \item 确定输入输出。
    \item 写例子。
    \item 正反面考虑。
    \item 大致思路,边界考虑。
    \item 验证。

\end{itemize} 

\section{顺序表和链表的比较}
\begin{itemize}
    \item 存储结构:顺序表是连续存储,链表是非连续存储。
    \item 空间利用率:顺序表可能会浪费空间,链表空间利用率高。
    \item 访问速度:顺序表支持随机访问,链表只能顺序访问。
    \item 前者适合频繁查找,后者适合频繁插入删除。
\end{itemize}

\section{链表算法提升}
\subsection{双链表}
每个节点配两个指针,分别指向前驱与后继。提高查找链表结点前驱的效率。
\subsection{静态链表}
使用数组存取一堆结点,结点之间的指针用数组位置取代。
\subsection{循环链表}
把尾节点指向首节点,做成单循环链表;
把尾的结点的后继指向首数据的结点,并把首数据的前驱指向尾结点,做成双循环链表。
\subsection{带头节点的链表(ppt中的小技巧,也叫哨兵节点)}
在链表的最前面加一个不存数据的节点,作为头节点。
\subsection{间接寻址}
这是对于顺序表的一个改版,生成的数列不直接存数据,而是存一个指向数据的指针。这个改
版的好处是插入、删除移动的只是这堆指针,而不是整个数据。(对于每个数据元素比较庞大时,这个确保
了移动量很小)。

\end{document}