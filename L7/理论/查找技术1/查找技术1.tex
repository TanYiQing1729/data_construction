\documentclass{article}
\usepackage[UTF8]{ctex} 
\usepackage{amsmath}
\usepackage{bm}
\usepackage{amssymb}
\usepackage{geometry}
\usepackage{graphicx}
\usepackage{listings}
\usepackage{tikz}
\usepackage{amsthm}
\newtheorem{theorem}{定理}
\usetikzlibrary{calc}
\usetikzlibrary{shapes.multipart}
\usetikzlibrary{arrows.meta, positioning, shapes.geometric}
\usetikzlibrary{decorations.pathreplacing, fit}
\geometry{a4paper, margin=1.5cm}

\title{查找技术1}
\author{Tan Yiqing}
\date{\today}

\begin{document}
\maketitle
    
    \begin{figure}[h]
        \centering
        \includegraphics[width=0.68\textwidth]{D:/program/data_construction/firefly/20251201171833_574_13.jpg}
    \end{figure}

\section{概念}
\indent 先介绍一些定义
\begin{enumerate}
    \item 关键码:识别一个记录的某个数据项
    \item 键码:关键码的值
    \item 主关键码:可以唯一标识一个记录的关键码
    \item 次关键码:不能唯一标识一个记录的关键码
    \item 静态查找:不涉及插入和删除操作的查找,线性表用的多
    \item 动态查找:涉及插入和删除操作的查找,树用的多
\end{enumerate}

查找算法的时间性能通过关键码的比较次数来衡量。具体来说,定义:\\
\indent 平均查找长度ASL(Average Search Length):表示在进行多次查找操作后,平均每次查找所需的关键码比较次数。
\[
    ASL=\sum_{i=1}^{n} P_i \cdot L_i
\]
其中$P_i$表示查找第i个记录的概率(与算法无关),$L_i$表示查找第i个记录所需的关键码比较次数(与算法有关),n表示线性表中的记录总数。\\
\indent ASL是衡量查找算法效率的重要指标,ASL越小,表示查找效率越高。

\section{线性表查找技术}
\subsection{顺序查找}
\indent 顺序查找是最简单的一种查找方法。它从线性表的第一个记录开始,依次将每个记录的关键码与待查找的关键码进行比较,
直到找到与待查找关键码相等的记录,或者搜索到线性表的末尾为止。
\subsubsection{算法描述}
没什么好描述的
\subsubsection{时间复杂度分析}
假设有n个记录,查找成功时,平均查找长度ASL为:
\[
    ASL_{success}=\frac{1+2+...+n}{n}=\frac{n+1}{2}
\]
查找不成功时,平均查找长度ASL为:
\[
    ASL_{failure}=\frac{n+1}{2}
\]
算法复杂度为O(n),太慢了,但是比较好实现。

\subsection{折半查找}
\subsubsection{使用条件}
折半查找要求线性表中的记录必须\pmb{按关键码有序},且采用\pmb{顺序存储}。
\subsubsection{算法描述}
每次将待查找的关键码与线性表中间位置的记录的关键码进行比较,
\begin{enumerate}
    \item 如果相等则查找成功;
    \item 如果待查找关键码小于中间位置记录的关键码,则在前半部分继续进行折半查找(high=mid-1);
    \item 否则在后半部分继续进行折半查找(low=mid+1)。
\end{enumerate}
如果low>high,则查找失败。

\subsubsection{折半查找判定树}
判定树其实是一棵二叉排序树,根节点表示待查找的关键码,左子树表示小于根节点的关键码,右子树表示大于根节点的关键码。\\

\begin{figure}[h]
    \centering
    \begin{tikzpicture}[level distance=1.2cm,
        level 1/.style={sibling distance=6cm},
        level 2/.style={sibling distance=3cm},
        level 3/.style={sibling distance=1.5cm}]
        \node {50}
            child {node {25}
                child {node {12}
                    child {node {6}}
                    child {node {18}}}
                child {node {37}
                    child {node {31}}
                    child {node {43}}}}
            child {node {75}
                child {node {62}
                    child {node {56}}
                    child {node {68}}}
                child {node {87}
                    child {node {81}}
                    child {node {93}}}};
    \end{tikzpicture}
    \caption{折半查找判定树示例}
\end{figure}


\subsubsection{时间复杂度分析}
可以从判定树的高度来分析时间复杂度。对于n个记录的线性表,折半查找的判定树高度为$\lfloor \log_2 n \rfloor + 1$。
因此,折半查找的时间复杂度为O(log n)。这是比顺序查找更高效的查找方法。

\section{树表查找技术}
\subsection{二叉排序树查找}
\subsubsection{二叉排序树}
\indent 二叉排序树(Binary Search Tree, BST)是一种特殊的二叉树,也称二叉查找树,满足以下性质:
\begin{enumerate}
    \item 或者是一个空树;
    \item 或者满足以下性质:
    \begin{enumerate}
        \item 若左子树非空,则左子树所有节点的关键码小于根节点的关键码。
        \item 若右子树非空,则右子树所有节点的关键码大于根节点的关键码。
        \item 左右子树也是二叉排序树。
    \end{enumerate}
\end{enumerate}
二叉排序树的中序遍历结果是一个有序的序列。

\subsubsection{二叉排序树的构建}
\indent 构建二叉排序树的过程如下,是比较简单的:
\begin{enumerate}
    \item 如果树为空,则将待插入的关键码作为根节点。
    \item 否则,比较待插入的关键码与当前节点的关键码:
    \begin{enumerate}
        \item 如果待插入关键码小于当前节点关键码,则递归地将其插入左子树。
        \item 如果待插入关键码大于当前节点关键码,则递归地将其插入右子树。
    \end{enumerate}
\end{enumerate}

\subsubsection{二叉排序树的删除}
\indent 二叉排序树的删除比较复杂,删除节点时需要考虑三种情况:
\begin{enumerate}
    \item 删除的节点是叶子节点,直接删除即可。
    \item 删除的节点只有一个子节点,将该节点的子节点连接到其父节点。
    \item 删除的节点有两个子节点,找到该节点的中序后继(右子树中最小的节点)或中序前驱(左子树中最大的节点),
    用该节点的关键码替换待删除节点的关键码,然后删除中序后继或前驱节点。
\end{enumerate}

\subsubsection{二叉排序树查找}
其实和折半查找差不多:
\begin{enumerate}
    \item 如果当前节点为空,则查找失败。
    \item 如果待查找关键码等于当前节点关键码,则查找成功。
    \item 如果待查找关键码小于当前节点关键码,则递归地在左子树中查找。
    \item 如果待查找关键码大于当前节点关键码,则递归地在右子树中查找。
\end{enumerate}

\subsubsection{时间复杂度分析}
取决于树的形状,在O(log n)到O(n)之间变化。理想情况下,树是平衡的,时间复杂度为O(log n)。
但在最坏情况下,树退化为链表,时间复杂度为O(n)。

\subsection{平衡二叉树查找}
\subsubsection{平衡二叉树}
\indent 很自然地,我们希望二叉排序树的时间复杂度贴近O(log n),为此引入平衡二叉树的概念。\\
\paragraph{平衡二叉树}
平衡二叉树(AVL树)是一种特殊的二叉排序树,满足以下性质:
\begin{enumerate}
    \item 或者空树;
    \item 或者具有以下性质:
    \begin{enumerate}
        \item 根结点的左子树和右子树的深度最多相差1.
        \item 左子树和右子树也是平衡二叉树。
    \end{enumerate}
\end{enumerate}
\paragraph{平衡因子}
\indent 左子树深度减去右子树深度。
\[
    BF(node)=height(node.left)-height(node.right)
\]
\paragraph{最小不平衡子树}
\indent 在平衡二叉树的构造过程中,以距离插入结点最近的,且平衡因子绝对值大于1的结点为根结点的子树,称为最小不平衡子树。

\indent 例如,依次插入 $1,2,3,4$ 到一棵初始为空的平衡二叉树(AVL树):

\begin{enumerate}
    \item 插入 $1$:树只有一个结点,无需调整。
    \item 插入 $2$:$2$ 作为 $1$ 的右孩子,树依然平衡。
    \item 插入 $3$:$3$ 作为 $2$ 的右孩子,此时 $1$ 的平衡因子变为 $-2$,失衡。
\end{enumerate}

\begin{center}
\begin{tikzpicture}[every node/.style={circle,draw,minimum size=7mm, font=\small}]
    \node (n1) at (0,0) {1};
    \node (n2) at (1.5,-1.2) {2};
    \node (n3) at (3,-2.4) {3};
    \draw (n1) -- (n2);
    \draw (n2) -- (n3);
    % 标注最小不平衡子树
    \draw[red,thick,rounded corners=6pt]
        ($(n1)+(-0.5,0.5)$) rectangle ($(n2)+(0.5,-0.5)$);
    \node[red, right=2.2cm of n1, font=\small] {以1为根的子树为最小不平衡子树};
\end{tikzpicture}
\end{center}

\noindent
此时,以结点 $1$ 为根的子树平衡因子绝对值大于 $1$,且它是距离插入点 $3$ 最近的失衡结点,
所以以 $1$ 为根的子树就是“最小不平衡子树”。只需对该子树做一次左旋,整棵树即可恢复平衡。

\paragraph{L层平衡二叉树的结点数上下界}
\begin{theorem}
    设一棵平衡二叉树(AVL 树)的高度为 $k$(共有 $k$ 层,根为第 1 层),记该高度下所有平衡二叉树中\pmb{结点数最少}的为 $M_k$。则:\\
    上界:任意高度为 $k$ 的二叉树,结点数不超过满二叉树:
    \[
        N_k \le 2^k - 1.
    \]
    下界:$M_k$ 满足
    \[
        M_1=1,\quad M_2=2,\quad M_k = M_{k-1} + M_{k-2} + 1\quad (k\ge 3),
    \]
    且
    \[
        M_k = F_{k+2} - 1,
    \]
    其中 $F_k$ 为 Fibonacci 数列,$F_1=1,F_2=1,F_k=F_{k-1}+F_{k-2}$。因此存在常数 $c>0$ 和
    \[
        \varphi=\frac{1+\sqrt{5}}{2},
    \]
    使得
    \[
        M_k = \Omega(\varphi^{\,k}).
    \]
\end{theorem}

\begin{proof}
\textbf{1. 上界} \\
高度为 $k$ 的任意二叉树,第 $i$ 层最多 $2^{i-1}$ 个结点:
\[
    N_k \le \sum_{i=1}^{k} 2^{i-1} = 2^k - 1,
\]
当树为满二叉树时取等号。

\medskip
\textbf{2. 下界的递推关系} \\
AVL 树中,每个结点左右子树高度差不超过 1,且左右子树本身也是 AVL 树。记高度为 $k$ 且结点数最少的 AVL 树为 $T_k$,结点数为 $M_k$。显然
\[
    M_1=1,\quad M_2=2.
\]

对 $k\ge 3$,设 $T_k$ 根结点左右子树高度分别为 $h_L,h_R$。则有
\[
    \max\{h_L,h_R\}=k-1,\quad |h_L-h_R|\le 1,
\]
因此 $\{h_L,h_R\}$ 只能为 $(k-1,k-1)$,$(k-1,k-2)$ 或 $(k-2,k-1)$。\\
为使整棵树结点数最少,应取高度更“不平衡”的情形,即 $(k-1,k-2)$(或对称的 $(k-2,k-1)$)。此时左右子树本身也应取各自高度下的最少结点数,故
\[
    \text{左子树结点数}=M_{k-1},\quad \text{右子树结点数}=M_{k-2}.
\]
再加上根结点 1 个:
\[
    M_k = M_{k-1} + M_{k-2} + 1,\quad k\ge 3.
\]

\medskip
\textbf{3. 与 Fibonacci 数列的关系} \\
Fibonacci 数列定义为
\[
    F_1=1,\ F_2=1,\ F_k=F_{k-1}+F_{k-2}\ (k\ge 3).
\]
声称
\[
    M_k = F_{k+2}-1,\quad k\ge 1.
\]
用归纳法证明:\\
当 $k=1,2$ 时:
\[
    M_1=1=F_3-1,\quad M_2=2=F_4-1.
\]
设对 $k-1,k-2$ 成立,即 $M_{k-1}=F_{k+1}-1,\ M_{k-2}=F_k-1$,则
\[
\begin{aligned}
    M_k &= M_{k-1}+M_{k-2}+1\\
        &= (F_{k+1}-1)+(F_k-1)+1\\
        &= F_{k+1}+F_k-1=F_{k+2}-1.
\end{aligned}
\]
故对一切 $k\ge 1$ 成立。

\medskip
\textbf{4. 渐近下界} \\
Fibonacci 数列有通项公式(Binet 公式):
\[
    F_k=\frac{\varphi^k-\psi^k}{\sqrt{5}},
\]
其中
\[
    \varphi=\frac{1+\sqrt{5}}{2},\quad \psi=\frac{1-\sqrt{5}}{2},\quad |\psi|<1.
\]
因此存在常数 $c_1,c_2>0$,对充分大的 $k$ 有
\[
    c_1\varphi^k \le F_k \le c_2\varphi^k,
\]
即 $F_k=\Theta(\varphi^k)$,特别地 $F_k=\Omega(\varphi^k)$。由
\[
    M_k = F_{k+2}-1
\]
可知,$M_k$ 与 $F_{k+2}$ 同阶,故
\[
    M_k = \Omega(\varphi^{k+2}) = \Omega(\varphi^k).
\]

综上,高度为 $k$ 的 AVL 树的最少结点数满足
\[
    M_k = F_{k+2}-1 = \Omega(\varphi^k),
\]
最多结点数为 $2^k-1$,命题得证。
\end{proof}

\paragraph{n个结点平衡二叉树的层数上下界}
\begin{theorem}
    设一棵平衡二叉树(AVL 树)有 $n$ 个结点,高度(层数)为 $k$。则其高度满足
    \[
        c_1\log n \ \le\ k\ \le\ c_2\log n
    \]
    其中 $c_1,c_2>0$ 为与 $n$ 无关的常数。也就是说,平衡二叉树的高度是 $k=\Theta(\log n)$。
\end{theorem}

\begin{proof}
记高度为 $k$ 的 AVL 树的\textbf{最少}结点数为 $M_k$,\textbf{最多}结点数为 $N_k$。

由前一条定理我们已经知道:
\[
    M_k = F_{k+2}-1,\quad M_k = \Omega(\varphi^{\,k}),\quad
    N_k \le 2^k - 1,
\]
其中 $\varphi=\dfrac{1+\sqrt{5}}{2}>1$,$F_k$ 为 Fibonacci 数列。

\medskip
\textbf{1. 利用下界 $M_k$ 求高度的上界}\\
对任意高度为 $k$ 的 AVL 树,有
\[
    n \;\ge\; M_k.
\]
而 $M_k=\Omega(\varphi^{\,k})$,即存在常数 $c>0$,以及充分大的 $k_0$,使得对 $k\ge k_0$:
\[
    M_k \;\ge\; c\,\varphi^{\,k}.
\]
于是
\[
    n \;\ge\; M_k \;\ge\; c\,\varphi^{\,k},
\]
从而
\[
    \varphi^{\,k} \;\le\; \frac{n}{c},
    \qquad
    k \;\le\; \log_{\varphi}\!\left(\frac{n}{c}\right)
    = O(\log n).
\]
这给出了 AVL 树高度关于结点数的一个\textbf{上界}:存在常数 $c_2>0$,使得
\[
    k \le c_2 \log n.
\]

\medskip
\textbf{2. 利用上界 $N_k$ 求高度的下界}\\
另一方面,对任意高度为 $k$ 的二叉树,有
\[
    n \;\le\; N_k \;\le\; 2^k - 1.
\]
因此
\[
    n+1 \;\le\; 2^k,
    \qquad
    k \;\ge\; \log_2(n+1).
\]
从而存在常数 $c_1>0$,使得
\[
    k \ge c_1 \log n.
\]

\medskip
\textbf{3. 合并两侧不等式}\\
综合 (1) 与 (2),得到:对结点数为 $n$ 的任意 AVL 树,其高度 $k$ 满足
\[
    c_1\log n \ \le\ k\ \le\ c_2\log n,
\]
即
\[
    k = \Theta(\log n).
\]
定理得证。
\end{proof}

\noindent 事实上,n个结点的平衡二叉树的高度下界更精确地为完全二叉树的情况:
\[
    k \ge \lfloor \log_2 n \rfloor + 1.
\]


\end{document}